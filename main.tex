%-----------------------------------------------------------------------------------------------------------------------------------------------%
%	The MIT License (MIT)
%
%	Copyright (c) 2021 Jitin Nair
%
%	Permission is hereby granted, free of charge, to any person obtaining a copy
%	of this software and associated documentation files (the "Software"), to deal
%	in the Software without restriction, including without limitation the rights
%	to use, copy, modify, merge, publish, distribute, sublicense, and/or sell
%	copies of the Software, and to permit persons to whom the Software is
%	furnished to do so, subject to the following conditions:
%	
%	THE SOFTWARE IS PROVIDED "AS IS", WITHOUT WARRANTY OF ANY KIND, EXPRESS OR
%	IMPLIED, INCLUDING BUT NOT LIMITED TO THE WARRANTIES OF MERCHANTABILITY,
%	FITNESS FOR A PARTICULAR PURPOSE AND NONINFRINGEMENT. IN NO EVENT SHALL THE
%	AUTHORS OR COPYRIGHT HOLDERS BE LIABLE FOR ANY CLAIM, DAMAGES OR OTHER
%	LIABILITY, WHETHER IN AN ACTION OF CONTRACT, TORT OR OTHERWISE, ARISING FROM,
%	OUT OF OR IN CONNECTION WITH THE SOFTWARE OR THE USE OR OTHER DEALINGS IN
%	THE SOFTWARE.
%	
%
%-----------------------------------------------------------------------------------------------------------------------------------------------%

%----------------------------------------------------------------------------------------
%	DOCUMENT DEFINITION
%----------------------------------------------------------------------------------------

% article class because we want to fully customize the page and not use a cv template
\documentclass[a4paper,12pt]{article}

%----------------------------------------------------------------------------------------
%	FONT
%----------------------------------------------------------------------------------------

% % fontspec allows you to use TTF/OTF fonts directly
% \usepackage{fontspec}
% \defaultfontfeatures{Ligatures=TeX}

% % modified for ShareLaTeX use
% \setmainfont[
% SmallCapsFont = Fontin-SmallCaps.otf,
% BoldFont = Fontin-Bold.otf,
% ItalicFont = Fontin-Italic.otf
% ]
% {Fontin.otf}

%----------------------------------------------------------------------------------------
%	PACKAGES
%----------------------------------------------------------------------------------------
\usepackage{url}
\usepackage{parskip} 	

%other packages for formatting
\RequirePackage{color}
\RequirePackage{graphicx}
\usepackage[usenames,dvipsnames]{xcolor}
\usepackage[scale=0.9]{geometry}

%tabularx environment
\usepackage{tabularx}

%for lists within experience section
\usepackage{enumitem}

% centered version of 'X' col. type
\newcolumntype{C}{>{\centering\arraybackslash}X} 

%to prevent spillover of tabular into next pages
\usepackage{supertabular}
\usepackage{tabularx}
\newlength{\fullcollw}
\setlength{\fullcollw}{0.47\textwidth}

%custom \section
\usepackage{titlesec}				
\usepackage{multicol}
\usepackage{multirow}

%CV Sections inspired by: 
%http://stefano.italians.nl/archives/26
\titleformat{\section}{\Large\scshape\raggedright}{}{0em}{}[\titlerule]
\titlespacing{\section}{0pt}{10pt}{10pt}

%for publications
%\usepackage[style=authoryear,sorting=ynt, maxbibnames=2]{biblatex}
\usepackage[style=numeric,sorting=anyt, maxbibnames=2]{biblatex}

%Setup hyperref package, and colours for links
\usepackage[unicode, draft=false]{hyperref}
\definecolor{linkcolour}{rgb}{0,0.2,0.6}
\hypersetup{colorlinks,breaklinks,urlcolor=linkcolour,linkcolor=linkcolour}
\addbibresource{citations.bib}
\setlength\bibitemsep{1em}

%for social icons
\usepackage{fontawesome5}

%debug page outer frames
%\usepackage{showframe}

% job listing environments
\newenvironment{jobshort}[2]
    {
    \begin{tabularx}{\linewidth}{@{}l X r@{}}
    \textbf{#1} & \hfill &  #2 \\[3.75pt]
    \end{tabularx}
    }
    {
    }

\newenvironment{joblong}[2]
    {
    \begin{tabularx}{\linewidth}{@{}l X r@{}}
    \textbf{#1} & \hfill &  #2 \\[3.75pt]
    \end{tabularx}
    \begin{minipage}[t]{\linewidth}
    \begin{itemize}[nosep,after=\strut, leftmargin=1em, itemsep=2pt,label=--]
    }
    {
    \end{itemize}
    \end{minipage}
    }



%----------------------------------------------------------------------------------------
%	BEGIN DOCUMENT
%----------------------------------------------------------------------------------------
\begin{document}

% non-numbered pages
\pagestyle{empty} 

%----------------------------------------------------------------------------------------
%	TITLE
%----------------------------------------------------------------------------------------

% \begin{tabularx}{\linewidth}{ @{}X X@{} }
% \huge{Your Name}\vspace{2pt} & \hfill \emoji{incoming-envelope} email@email.com \\
% \raisebox{-0.05\height}\faGithub\ username \ | \
% \raisebox{-0.00\height}\faLinkedin\ username \ | \ \raisebox{-0.05\height}\faGlobe \ mysite.com  & \hfill \emoji{calling} number
% \end{tabularx}

\begin{tabularx}{\linewidth}{@{} C @{}}
\Huge{Luke Martin Vaughan} \\[7.5pt]
%\href{https://github.com/username}{\raisebox{-0.05\height}\faGithub\ username} \ $|$ \ 
%\href{https://linkedin.com/in/username}{\raisebox{-0.05\height}\faLinkedin\ username} \ $|$ \ 
%\href{https://mysite.com}{\raisebox{-0.05\height}\faGlobe \ mysite.com} \ $|$ \ 
%\href{mailto:luke.vaughan@okstate.com}{\raisebox{-0.05\height}\faEnvelope \ luke.vaughan@okstate.edu} \ $|$ \ 
%\href{tel:+19188991795}{\raisebox{-0.05\height}\faMobile \ +1 918-899-1795} \\
Oklahoma State University \ $|$ \
ATLAS Collaboration \ $|$ \
luke.vaughan@okstate.edu \\
\end{tabularx}

%----------------------------------------------------------------------------------------
% SUMMARY
%----------------------------------------------------------------------------------------
%Interests/ Keywords/ Summary
\section{Research Interest}
Experimental high energy particle physics; using machine learning to model complex data with focused
concentration on Higgs physics, pileup mitigation, top quark polarimetry, boosted jet tagging, data
reconstruction, and calibration techniques.

%----------------------------------------------------------------------------------------
%	EDUCATION
%----------------------------------------------------------------------------------------
\section{Education}
\begin{tabularx}{\linewidth}{@{}l X@{}}	
May 2026 (Expected) & PhD Physics at Oklahoma State University \hfill \normalsize (GPA: 3.9) \\

May 2024 & M.S. Physics at Oklahoma State University \hfill (GPA: 4.0) \\ 

May 2021 & B.S. Physics at Oklahoma State University \hfill (GPA: 4.0) \\ 

May 2021 & B.S. Aerospace Engineering at Oklahoma State University \hfill (GPA: 4.0) \\ 

May 2021 & B.S. Mechanical Engineering at Oklahoma State University \hfill (GPA: 4.0) \\ 

\end{tabularx}

%----------------------------------------------------------------------------------------
% EXPERIENCE SECTIONS
%----------------------------------------------------------------------------------------
\section{Research Experience}

\begin{jobshort}{ATLAS Analysis: Hadronic Higgs in the VH production mode}{Spring 2023 - Fall 2023}
I investigated adding a new low pT region with high statistics in the 1-Lepton channel to better constrain backgrounds.
Due to the high statistics, this region caused significant pulls on nuisance parameters which I mitigated by implementing a new set of decorrelated parameters.
By including this region the Z boson sensitivity increased by nearly 20\% which allowed for $5.2\sigma$ on the $VZ, Z\rightarrow c\bar{c}$ process \cite{VH}.
\end{jobshort}

\begin{jobshort}{ATLAS Analysis: Hadronic Higgs in VBF production mode}{Fall 2023 - Current}
Hadronic Higgs decays in VBF production mode are limited by statistics and subtracting the large non-resonant QCD background poses a significant challenge.
I designed and optimized a framework to train adversarial NNs to define signal regions with minimal background sculpting.
The final fit combination that I performed led to the strongest results to date regarding $VBF,H \rightarrow b\bar{b}$ significance and $VH+VBF,H \rightarrow c\bar{c}$ limits\cite{VBF}.
\end{jobshort}

\begin{jobshort}{ATLAS Analysis: GN2X $t\bar{t}$ Calibration}{Spring 2023 - Spring 2024}
GN2X is a novel boosted jet tagging algorithm to distinguish between Hbb, Hcc, top, and QCD.
I contributed to the evaluation and calibration of GN2Xv01 on semi-leptonic ttbar events.
I performed a likelihood fit to derive scale factors to quantify the mistag rate for tagged $t\bar{t}$ events.
\end{jobshort}

\begin{jobshort}{R\&D Project: Pileup Mitigation using Graph \& Attention NN}{Fall 2023 - Current}
Pileup mitigation will be a crucial problem for the High-Lumi LHC, and implementing state of the art algorithms is necessary to maximize the LHC discovery potential.
I developed graph and attention based neural networks in PyTorch to encode events and extract event-wide correlations.
This approach provides continuous energy and mass corrections at the jet level and demonstrates significant improvement to diHiggs reconstruction in the HL-LHC environment\cite{PAKDD}.
For this project, I developed a simulation framework using MadGraph + Pythia for HPC environment, that scales efficiently to $\langle \mu \rangle=200$.
\end{jobshort}

\begin{jobshort}{R\&D Project: Hadronic Top Quark Polarimetry using ML}{Fall 2024 - Current}
Top quark entanglement can be studied at high energy through the spin correlations of decay products.
Traditionally, only the leptonic decays of the W boson are studied, however I have developed an attention NN to analyze the hadronic decays of the W boson which enables top entanglement analyses to cover a broader phase space with more statistics.
\end{jobshort}

%----------------------------------------------------------------------------------------
%	PUBLICATIONS
%----------------------------------------------------------------------------------------
\section{Publications}
\begin{refsection}[citations.bib]
\nocite{*}
\printbibliography[heading=none]
\end{refsection}

%----------------------------------------------------------------------------------------
%	SKILLS
%----------------------------------------------------------------------------------------
\section{Skills}
\begin{tabularx}{\linewidth}{@{}l X@{}}
Programming Languages &  \normalsize{Python, C/C++, Bash, GNU Linux}\\
Monte-Carlo Simulation &  \normalsize{MadGraph, Pythia, Delphes, FastJet, ATLAS Software}\\
Data Analysis &  \normalsize{ROOT, Numpy, Awkward, Matplotlib, PyTorch, TensorFlow, Scikit-Learn}\\
Computing &  \normalsize{Parallel Computing, Accelerated Computing, NFS Shares, Homelab SysAdmin}\\
\end{tabularx}

%----------------------------------------------------------------------------------------
%	CONFERENCES
%----------------------------------------------------------------------------------------
\section{Conference Presentations}
\begin{jobshort}{Lepton-Photon 2025}{\href{https://indico.cern.ch/event/1493037/timetable/?view=standard\#54-measurements-of-higgs-boson}{Madison, Wisconsin}}
\textit{Measurements of Higgs Bosons Decaying to Bottom and Charm Quarks from Vector Boson Fusion Production with the ATLAS Experiment}
\end{jobshort}

\begin{jobshort}{PAKDD 2025}{\href{https://pakdd2025.org/detailed-program/day-2/}{Sydney, Australia June 2025}}
\textit{PileUp Mitigation at the HL-LHC Using Attention for Event-Wide Context} arXiv \href{https://arxiv.org/abs/2503.02860}{2503.02860}
\end{jobshort}

\begin{jobshort}{APS 2025 Mini-Symposium: Top Quark Physics and Beyond}{\href{https://schedule.aps.org/smt/2025/events/APR-G10}{Anaheim, CA April 2025}}
\textit{Boosted $X\rightarrow b\bar{b}$ tagger calibration using semi-leptonic ttbar events collected with the ATLAS detector}
\end{jobshort}

\begin{jobshort}{APS 2025 AI/ML Poster Session}{\href{https://schedule.aps.org/smt/2025/events/MAR-H00/323}{Anaheim, CA April 2025}}
\textit{Pileup Mitigation at the High-Luminosity LHC using Attention Neural Networks}
\end{jobshort}

%----------------------------------------------------------------------------------------
%	WORKSHOPS
%----------------------------------------------------------------------------------------
\section{Workshops}
\begin{jobshort}{ATLAS $8^{th}$ Machine Learning Workshop}{\href{https://indico.cern.ch/event/1488957/}{CERN March 2025}}
Presented work on Attention Neural Networks for Jet energy and mass corrections for High-Lumi LHC.
\end{jobshort}

\begin{jobshort}{CoDaS-HEP Sixth Computational and Data Science School for HEP}{\href{https://indico.cern.ch/event/1422680/}{Princeton July 2024}}
Learned Awkward arrays in depth from the project developers. Introduced to high performance parallel computation.
\end{jobshort}

\begin{jobshort}{How to do ATLAS Analysis - a hands on Tutorial}{\href{https://indico.cern.ch/event/1309794/}{SLAC October 2023}}
Gained knowledge of ATLAS analysis tools such as AnalysisBase and how to apply them for general analysis.
\end{jobshort}

\begin{jobshort}{US ATLAS Machine Learning Training}{\href{https://indico.cern.ch/event/1264566/}{Lawrence Berkeley National Lab July 2023}}
Discussed application of various models in physics: MLP, Convolutional, Graph, Attention, Adversarial, Generative, Normalizing Flows, Invertible.
\end{jobshort}

%----------------------------------------------------------------------------------------
%	LEADERSHIP AND SERVICE
%----------------------------------------------------------------------------------------
\section{Leadership and Service}
\begin{jobshort}{President of Physics Graduate Student Association at OSU}{Summer 2024-Spring 2025}
Hosted events and meetings for the benefit of the graduate student population. Organized a poster symposium for graduate students to share and advertise research to new students. Represented the Physics Graduate Student body at the college-wide Student Government Association and applied for funding from the College.
\end{jobshort}

%----------------------------------------------------------------------------------------
%	TEACHING AND OUTREACH
%----------------------------------------------------------------------------------------
\section{Teaching and Outreach}
\begin{jobshort}{Mentor for REU Summer Students}{Summer 2025}
Aided upper-level undergraduate students with getting started with HEP-Ex research. Introduce them to software tools in HEP which include: simulation, preprocessing, plotting, and physical interpretation.
\end{jobshort}

\begin{jobshort}{Teaching Assistant PHYS 1114 Labs and Recitations}{Spring 2019-Fall 2023}
Guide students through introductory mechanics labs: gravity, springs, pendulums, etc.
\end{jobshort}

\begin{jobshort}{Tour Guide OSU Physics Department}{2023-Current}
Lead tours for incoming prospective students to show them various physics labs and motivate them to study physics.
\end{jobshort}

\begin{jobshort}{High School Talks}{2025}
Invited to speak to students at Bishop Kelley High School to encourage science, robotics, and STEM involvement.
\end{jobshort}

\vfill
\center{\footnotesize Last updated: \today}

\end{document}
